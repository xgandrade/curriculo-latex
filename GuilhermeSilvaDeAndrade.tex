%-------------------------
% Resume in Latex
% Author: Guilherme Silva de Andrade
% Based off of: https://github.com/sb2nov/resume
% License: MIT
%------------------------

\documentclass[letterpaper,11pt]{article}

\usepackage{latexsym}
\usepackage[empty]{fullpage}
\usepackage{titlesec}
\usepackage{marvosym}
\usepackage[usenames,dvipsnames]{color}
\usepackage{verbatim}
\usepackage{enumitem}
\usepackage[hidelinks]{hyperref}
\usepackage{fancyhdr}
\usepackage[portuguese]{babel}
\usepackage{tabularx}
\usepackage{fontspec}
\usepackage{ragged2e}
\setmainfont{Latin Modern Sans}

%----------FONT OPTIONS----------
\pagestyle{fancy}
\setlength{\footskip}{10pt}
\fancyhf{}
\fancyfoot{}
\renewcommand{\headrulewidth}{0pt}
\renewcommand{\footrulewidth}{0pt}

% Adjust margins
\addtolength{\oddsidemargin}{-0.5in}
\addtolength{\evensidemargin}{-0.5in}
\addtolength{\textwidth}{1in}
\addtolength{\topmargin}{-.5in}
\addtolength{\textheight}{1.0in}

\urlstyle{same}

\raggedbottom
\raggedright
\setlength{\tabcolsep}{0in}

% Sections formatting
\titleformat{\section}{
  \vspace{-4pt}\bfseries\raggedright\large
}{}{0em}{}[\color{black}\titlerule \vspace{-5pt}]

%-------------------------
% Custom commands
\newcommand{\resumeItem}[1]{
  \item\small{
    {#1 \vspace{-2pt}}
  }
}

\newcommand{\resumeSubheading}[4]{
  \vspace{-2pt}\item
    \begin{tabular*}{0.97\textwidth}[t]{l@{\extracolsep{\fill}}r}
      \textbf{#1} & #2 \\
      \textit{\small#3} & \textit{\small #4} \\
    \end{tabular*}\vspace{-7pt}
}

\newcommand{\resumeSubSubheading}[2]{
    \item
    \begin{tabular*}{0.97\textwidth}{l@{\extracolsep{\fill}}r}
      \textit{\small#1} & \textit{\small #2} \\
    \end{tabular*}\vspace{-7pt}
}

\newcommand{\resumeProjectHeading}[2]{
    \item
    \begin{tabular*}{0.97\textwidth}{l@{\extracolsep{\fill}}r}
      \small#1 & #2 \\
    \end{tabular*}\vspace{-7pt}
}

\newcommand{\resumeSubItem}[1]{\resumeItem{#1}\vspace{-4pt}}

\renewcommand\labelitemii{$\vcenter{\hbox{\tiny$\bullet$}}$}

\newcommand{\resumeSubHeadingListStart}{\begin{itemize}[leftmargin=0.15in, label={}]}
\newcommand{\resumeSubHeadingListEnd}{\end{itemize}}
\newcommand{\resumeItemListStart}{\begin{itemize}}
\newcommand{\resumeItemListEnd}{\end{itemize}\vspace{-5pt}}

%-------------------------------------------
%%%%%%  RESUME STARTS HERE  %%%%%%%%%%%%%%%%%%%%%%%%%%%%

\begin{document}

%----------HEADING----------
\begin{center}
    \textbf{\Huge \bfseries Guilherme Silva de Andrade} \\ \vspace{1pt}
    \small São Paulo, SP \ $|$ \ (11) 95441-2054 \ $|$ 
    \href{mailto:guilherme\_gsa@live.com}{\underline{guilherme\_gsa@live.com}} $|$ 
    \href{https://www.linkedin.com/in/xgandrade}{LinkedIn} \ $|$ 
    \href{https://github.com/xgandrade}{GitHub}
\end{center}

%----------- RESUME -----------
\section{Resumo profissional}
\justifying
Engenheiro de Software com mais de 8 anos de experiência em TI, especializado em desenvolvimento backend (.NET/C\#) e microsserviços. Tenho experiência em ambientes ágeis, trabalhando em equipes multidisciplinares e com foco em entregas contínuas e sustentáveis. Sou familiarizado com práticas de escalabilidade, resiliência e observabilidade, especialmente em nuvem (Azure/AWS). Valorizo a clareza no código, a entrega de soluções de alta qualidade e o bom relacionamento entre times e stakeholders, sempre buscando a evolução contínua dos processos e boas práticas de desenvolvimento.

%-----------PROGRAMMING SKILLS-----------
\section{Habilidades técnicas}
 \begin{itemize}[leftmargin=0.15in, label={}]
    \small{\item{
        \textbf{Linguagens:} C\#, JavaScript, Bash, Python \\
        \textbf{Frameworks:} .NET (Core/Framework), ASP.NET, Entity Framework, LINQ, NestJS, Node.js, Angular, React \\
        \textbf{Cloud:} Azure, AWS (App Services, Lambda, S3) \\
        \textbf{Banco de Dados:} SQL Server, PostgreSQL \\
        \textbf{Mensageria:} RabbitMQ, SQS \\
        \textbf{Ferramentas:} Git, Docker/Podman, Azure DevOps, Visual Studio \\
        \textbf{Práticas:} CI/CD, Testes automatizados, Clean Code, Design Patterns
    }}
 \end{itemize}

%----------- EDUCATION -----------
\section{Educação}
  \resumeSubHeadingListStart
    \resumeSubheading
      {USP/ESALQ (Cursando)}{São Paulo, SP}
      {MBA em Engenharia de Software}{Mai 2024 -- Dez 2025}
    \resumeSubheading
      {Universidade Nove de Julho}{São Paulo, SP}
      {Bacharelado em Ciência da Computação}{Fev 2017 -- Dez 2020}
  \resumeSubHeadingListEnd

%-----------EXPERIENCE-----------
\section{Experiência Profissional}
  \resumeSubHeadingListStart

    \resumeSubheading
      {Engenheiro de Software Sênior}{Out 2024 -- Atual}
      {TOTVS SA}{São Paulo, SP}
    \\[8pt]
    \justifying
    Atuação como referência técnica em soluções backend, contribuindo para a evolução de sistemas distribuídos em ambientes cloud. Responsável pelo desenvolvimento de soluções escaláveis com foco em mensageria assíncrona e serverless, além de apoiar tecnicamente o time e participar de definições arquiteturais.
      \resumeItemListStart
        \resumeItem{Desenvolvimento backend com C\#, integração com RabbitMQ e SQS;}
        \resumeItem{Apoio técnico a desenvolvedores juniores e revisão de código;}
        \resumeItem{Desenvolvimento de soluções serverless (AWS Lambda, Azure Functions);}
        \resumeItem{Participação ativa em decisões técnicas e definição de arquitetura;}
        \resumeItem{Práticas ágeis (Scrum), versionamento com Git e CI/CD com Azure DevOps;}
      \resumeItemListEnd
      
    \resumeSubheading
      {Engenheiro de Software Pleno}{Jul 2024 -- Out 2024}
      {LAQUS SA}{São Paulo, SP}
      \\[8pt]
      \justifying
      Na Laqus, atuei como engenheiro de software fullstack contribuindo para a evolução dos projetos core da plataforma. Participei ativamente das decisões técnicas e arquiteturais, especialmente em temas ligados à estrutura financeira e integração de novos instrumentos, como o CDCA (Certificado de Direito Creditório do Agronegócio), além de melhorias na fase final da escrituração de operações. \\
      \justifying
      Meu dia a dia envolve desenvolvimento backend com C\# (.NET Core 6) e NestJS (Node/TypeScript), e frontend com Angular, dentro de uma arquitetura baseada em MicroFrontends (MFE). Além disso, também participei de iniciativas voltadas à confiabilidade e performance das aplicações, colaborando com práticas de SRE (Site Reliability Engineering). \\
      \justifying
      Na parte de infraestrutura, atuo com recursos da AWS (EC2, S3, CloudWatch), uso RabbitMQ e SQS para mensageria assíncrona e Grafana para observabilidade. Trabalho com PostgreSQL como banco de dados relacional e contribuo com a estabilidade e evolução contínua do ambiente de produção, garantindo qualidade e disponibilidade dos serviços da plataforma.

    \resumeSubheading
      {Engenheiro de Software Pleno}{Jan 2023 -- Jun 2024}
      {B3 SA}{São Paulo, SP}
      \\[8pt]
      \justifying
      Atuei como engenheiro de software em projetos corporativos estratégicos, sendo ponto focal técnico em iniciativas que envolviam desde a concepção até a sustentação de soluções. Minha atuação envolveu o desenvolvimento de provas de conceito (POCs), desenho de arquiteturas, análise de requisitos junto a parceiros de negócio e interlocução com áreas cross da companhia. \\
      \justifying
      No dia a dia, trabalhei com automações em .NET Core e scripts em batch/shell, com controle de execução via Control-M e Azure Databricks. Utilizei SQL Server como banco de dados principal e Bitbucket para versionamento dos projetos. Também participei de atividades relacionadas à sustentação de sistemas legados, bem como na criação e manutenção de documentações técnicas.
      \resumeItemListStart
        \resumeItem{Ponto focal técnico para automações e arquitetura de projetos.}
        \resumeItem{Uso de .NET, SQL Server, Azure (B2C, ADF, MS Entra ID) e Control-M.}
        \resumeItem{Participação em projetos como Talkdesk, Market Surveillance e monitoramento de mídias sociais.}
        \resumeItem{Sustentação e evolução com escrita de requisitos dos sistemas MRP (Mecanismo de Ressarcimento de Prejuízos) e OnBase (Portal BSM).}
        \resumeItem{Participação no desenvolvimento em nuvem de soluções para o Market Surveillance}
      \resumeItemListEnd

    \resumeSubheading
      {Engenheiro de Software Jr}{Fev 2017 -- Dez 2022}
      {TOTVS SA}{São Paulo, SP}
      \\[8pt]
      \justifying
      Atuei em diferentes frentes de desenvolvimento, com crescimento técnico e aumento de responsabilidades ao longo dos anos. Iniciei na torre MPN com foco em sustentação e evolução dos produtos legados Fly01 Manufatura (ADVPL) e Fly01 Vitrine (Genexus), atuando de forma autônoma na manutenção e atendimento a requisitos legais. \\
      \justifying
      A partir de 2021, integrei o time do Eleve Gestão, trabalhando com C\#, ASP.NET MVC, JavaScript e Node.js, além de T-SQL e Entity Framework, com foco em sustentação, inovação e performance de módulos como financeiro, estoque e tributação. \\
      \justifying
      Utilizei metodologias ágeis (Scrum), Azure DevOps para integração contínua, e RabbitMQ para mensageria entre módulos do ERP.
      \resumeItemListStart
        \resumeItem{Sustentação e evolução de sistemas legados (ADVPL e Genexus);}
        \resumeItem{Desenvolvimento de novas funcionalidades em aplicações web com C\#, JavaScript e Node.js;}
        \resumeItem{Correção de falhas e melhorias técnicas em código e banco de dados (SQL Server);}
        \resumeItem{Atuação funcional nos domínios de fiscal, financeiro e estoque;}
        \resumeItem{Atuação com arquitetura MVC, mensageria assíncrona e repositórios Git;}
        \resumeItem{Participação em rituais ágeis e revisões de código;}
      \resumeItemListEnd

  \resumeSubHeadingListEnd

%----------- PROJETOS -----------
\section{Projetos}
  \resumeSubHeadingListStart

    \resumeProjectHeading
      {\textbf{\href{https://tdn.totvs.com/display/public/CMNET/BACK+SaaS}{BackSaaS - backoffice Hotelaria (TOTVS)}} $|$ \emph{C\#, JS, RabbitMQ, SQS}}{Out 2024 -- Atual}
      \resumeItemListStart
        \resumeItem{Desenvolvimento da plataforma BackSaaS, focada em soluções para o setor de hotelaria, com integração entre sistemas de gestão (PMS) e pontos de venda (POS).}
        \resumeItem{Implementação de serviços backend em C\# e integrações serverless com AWS Lambda (Node.js) e Azure Functions (C\#).}
        \resumeItem{Utilização de mensageria assíncrona com RabbitMQ e SQS para comunicação entre serviços distribuídos.}
      \resumeItemListEnd

    \resumeProjectHeading        
    {\textbf{\href{https://www.bsmsupervisao.com.br/w/curso-sobre-market-surveillance-e-compliance-em-parceria-com-o-insper}{Market Surveillance}} $|$ \emph{Python, Azure Databricks, Trino}}{Jan 2023 -- Jun 2024}
      \resumeItemListStart
        \resumeItem{Desenvolvimento em nuvem do sistema Market Surveillance, utilizado para o monitoramento contínuo de operações atípicas no mercado.}
        \resumeItem{Automação de rotinas da Engenharia de Dados com Python e workflows no Azure Databricks para consumo de dados via REST.}
      \resumeItemListEnd

  \resumeSubHeadingListEnd

%----------- CERTIFICAÇÕES -----------
\section{Certificações}
\resumeItemListStart
  \resumeItem{\href{https://udemy-certificate.s3.amazonaws.com/pdf/UC-19fffb71-8789-4998-bf01-02d65251b3b7.pdf}{C\# COMPLETO Programação Orientada a Objetos + Projetos – Udemy – 2025}}
  \resumeItem{\href{https://cursos.alura.com.br/degree/certificate/c9a9f74e-4f7c-42e0-8b24-7dd3c9902c13?lang=pt_BR}{Formação ASP.Net Core: crie aplicações com C\#, .NET, Entity Framework e LINQ – Alura – 2025}}
  \resumeItem{\href{https://www.hackerrank.com/certificates/iframe/18c4a555d0e1}{SQL (Basic) Certificate – Hackerrank – 2025}}
\resumeItemListEnd

\end{document}
